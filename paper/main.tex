\documentclass{article}

% if you need to pass options to natbib, use, e.g.:
% \PassOptionsToPackage{numbers, compress}{natbib}
% before loading nips_2016
%
% to avoid loading the natbib package, add option nonatbib:
% \usepackage[nonatbib]{nips_2016}

%\usepackage{nips_2016}

% to compile a camera-ready version, add the [final] option, e.g.:
\usepackage[final]{nips_2016}

\usepackage[utf8]{inputenc} % allow utf-8 input
\usepackage[T1]{fontenc}    % use 8-bit T1 fonts
\usepackage{hyperref}       % hyperlinks
\usepackage{url}            % simple URL typesetting
\usepackage{booktabs}       % professional-quality tables
\usepackage{amsfonts, amssymb}       % blackboard math symbols
\usepackage{nicefrac}       % compact symbols for 1/2, etc.
\usepackage{microtype}      % microtypography
\usepackage{bm}
\usepackage{amsmath}
\usepackage{algorithm}
\usepackage{algorithmic}
\usepackage{subfigure}
\usepackage{wrapfig}

\newtheorem{theorem}{\bf{Theorem}}
\newtheorem{remark}{\bf{Remark}}

\usepackage{color}
\newcommand{\yongz}[1]{{\color{blue}{\bf\sf[#1]}}}
\newcommand{\qiu}[1]{{\color{green}{\bf\sf[#1]}}}

\usepackage{tikz}
\usetikzlibrary{fit,positioning,arrows,decorations.markings}
\setlength{\bibsep}{0.2ex}


\title{Distributed Publicly Verifiable Randomness without Trusted Third Party}

% The \author macro works with any number of authors. There are two
% commands used to separate the names and addresses of multiple
% authors: \And and \AND.
%
% Using \And between authors leaves it to LaTeX to determine where to
% break the lines. Using \AND forces a line break at that point. So,
% if LaTeX puts 3 of 4 authors names on the first line, and the last
% on the second line, try using \AND instead of \And before the third
% author name.

\author{
  Feiyang~Qiu\thanks{Use footnote for providing further
    information about author (webpage, alternative
    address)---\emph{not} for acknowledging funding agencies.} \\
  Department of Computer Science\\
  Tsinghua University\\
  Beijing, PRC 100084 \\
  \texttt{qfy14@mails.tsinghua.edu.cn} \\
  \thanks{hello} \\
  %% \And
  %% Coauthor \\
  %% Affiliation \\
  %% Address \\
  %% \texttt{email} \\
  %% \And
  %% Coauthor \\
  %% Affiliation \\
  %% Address \\
  %% \texttt{email} \\
}

\begin{document}

\maketitle
\begin{abstract}
  We present XXXX , a new XXXX.\cite{MMD:uai2015(example)}
\end{abstract}

\section{Introduction}



\yongz{problem. why it is important. our solution, contribution and paper layout.}

\section{Related Work}
\yongz{paper review:  how is our algorithm related to and different from the others, 
including the problem settings, algorithm differences. e.g. with comparison to Dfinity}

\section{Background}

\subsection{Shamir Distributed Secret-Sharing}

\subsection{ECDSA Schemes}

\subsection{Insecure Approches to Public Randomness}


\yongz{list technical component we used in this paper}

\section{Problem and Settings}

\subsection{Formal Model}

\paragraph{Execution model.} 

\begin{itemize}

\item Pemissionless blockchain.

\item Corruption. 

\end{itemize}



\subsection{Assumptions}

\section{Protocol}

\subsection{Our Protocol in A Nutshell}

\subsection{Algorithm Details}

\subsection{Analysis}

\subsection{Practical Implementation Concerns}


\section{Experiments}
g
\section{Conclusion}



\bibliography{main_ref}
\bibliographystyle{plain}

\appendix
\section{Implementation Details}
\section{Additional Results}

\end{document}

